\chapter{Background e Stato dell'Arte}\label{cap:Background}

\section{Educazione digitale}
In una società come quella odierna, ormai completamente immersa nella tecnologia, è impensabile che anche il mondo dell'istruzione non ne sia fortemente influenzato. 
Gli stessi studenti, che al giorno d'oggi sono tutti nativi digitali, stanno crescendo accompagnati da dispositivi tecnologici sempre più avanzati e da un'ampia varietà
di applicazioni e servizi digitali. In questo contesto, si ritiene di fondamentale importanza che la scuola, il centro di gravità della vita di ogni studente, sia in grado 
di fornire ai giovani tutti gli strumenti che servono non soltanto a comprendere il mondo digitale che hanno tra le mani, ma anche a saperne fare un uso responsabile nonché
fruttuoso per quanto riguarda la propria crescita personale e professionale. 

L'educazione digitale, dunque, non può più essere considerata un'attività opzionale da parte delle scuole e dei docenti, ma piuttosto un necessario investimento per fare in modo 
che le nuove generazioni siano in grado di sfruttare al massimo le opportunità che la tecnologia può offrire. 
Non solo: in realtà, l'educazione digitale è un argomento che copre una vasta gamma di ambiti di applicazione.
Si pensi soltanto alla sicurezza online e alla privacy: gli utenti di Internet devono essere informati sulle minacce digitali e imparare
come proteggere se stessi e i propri dati sensibili. 
In secondo luogo, l'educazione digitale si estende alla promozione di un comportamento etico e responsabile online, incoraggiando
l'empatia, il rispetto e la tolleranza nelle interazioni sociali. 
Per quanto riguarda, invece, lo sviluppo di competenze e \textit{soft skills} ormai indispensabili nella vita personale e lavorativa
di ogni individuo, è importante che sin dagli anni scolastici venga promosso e incentivato il pensiero critico, il \textit{problem solving} e la creatività, incoraggiando gli individui a
utilizzare gli strumenti digitali per esprimere idee, creare contenuti originali e collaborare online. Questa visione educativa non solo promuove lo sviluppo delle competenze digitali, ma anche una partecipazione
attiva nella società digitale in continua evoluzione.

\section{Pensiero computazionale}
Con l'espressione "pensiero computazionale" si fa riferimento ad un concetto che negli ultimi decenni ha guadagnato un'importanza sempre maggiore nell'ambito dell'educazione digitale. Questo termine si compone di due parole chiave: 
il primo, "pensiero", si riferisce alla caratteristica che rende l'essere umano unico rispetto a tutti gli altri esseri viventi, ovvero la capacità di ragionare, analizzare dati e situazioni e 
risolvere problemi. Il secondo termine, "computazionale", deriva dall'inglese \textit{to compute}, che può essere tradotto come "calcolare". Questi due termini, insieme, definiscono un concetto 
che molti, al giorno d'oggi, ritengono sia considerabile come la quarta abilità fondamentale per l'apprendimento, insieme alla lettura, alla scrittura e all'aritmetica: \textbf{pensare come un computer}.
Tuttavia, questa speciale competenza non nasce in concomitanza con l'utilizzo del computer, ma si tratta di processi mentali molto più antichi e sottili che non riguardano soltanto il mondo 
dell'informatica, ma possono essere applicati a qualsiasi contesto della vita quotidiana. 

\subsubsection{Seymour Papert}\label{papert}

\begin{multicols}{2}
    If children really want to learn something, and have the opportunity to learn it in use, they
    do so even if the teaching is poor. For example, many learn
    difficult video games with no professional teaching at all! Others
    use Nintendo's system of telephone hot lines or read magazines
    on strategies for games to find the kind of advice for video games
    that they would get from a teacher if this were a school subject.\newline
    \\
    Se i bambini vogliono davvero imparare qualcosa e hanno l'opportunità di imparare facendo, loro lo fanno anche se l'insegnamento è di scarsa qualità. Per esempio, molti imparano
    videogiochi difficili senza alcun insegnamento professionale! Altri
    utilizzano il sistema di Nintendo di linee telefoniche o leggono riviste
    sulle strategie per i videogiochi, il tipo di consigli
    che avrebbero ottenuto da un insegnante se questa fosse una materia scolastica.
  \end{multicols}

  \footnote[1]{Seymour Papert, \textit{The Children's Machine: Rethinking School in the Age of the Computer}, 1993.}

Il primo ad avere utilizzato il termine "Pensiero computazionale" fu \textbf{Seymour Papert}, matematico, informatico e pedagogista sudafricano naturalizzato statunitense. 
Papert può essere considerato uno dei padri dell'informatica educativa, ovvero di quella branca dell'informatica che si occupa di studiare e sviluppare metodi e strumenti per l'insegnamento e l'apprendimento in un nuovo contesto digitale.

Papert fu allievo di Jean Piaget – psicologo, biologo, pedagogista e filosofo svizzero – nonché il fondatore dell'epistemologia genetica, un settore della psicologia che
si occupa dello studio dei processi mentali e cognitivi messi in atto dal cervello umano al momento dell'apprendimento di nuove conoscenze. In particolare, Piaget viene ricordato per le sue teorie
costruttiviste, fortemente contrapposte ad un'altra tipologia di approccio didattico, l'istruzionismo. 

La \textbf{filosofia istruzionista} si basa sulla convinzione che la conoscenza e le competenze devono essere trasmesse dall'insegnante allo studente, 
il quale assume una posizione piuttosto passiva in un contesto dove altri strumenti
– ad esempio il computer – non sono altro che un semplice supporto per l'apprendimento. 

La \textbf{teoria costruttivista} di Piaget, al contrario, si incentra soprattutto su una visione di scuola dove 
i discenti sono attori attivi e partecipativi nel loro processo di apprendimento, poiché costruiscono e raffinano in modo autonomo e incrementale le proprie conoscenze e competenze.
In particolare Papert apprezzò moltissimo il lavoro di Piaget sul collegamento intrinseco che, secondo Piaget, esiste tra l'intelligenza – o, per meglio dire, capacità cognitiva – e l'abilità dell'individuo di adattarsi efficacemente all'ambiente sociale e fisico in cui 
si trova. In particolar modo, Papert si dedicò a consolidare il costruttivismo di Piaget in una sua versione più specifica, il \textbf{costruzionismo}. 

Papert, dall'incontro con le teorie di Piaget, comprese che il processo di appredimento può essere esplicato come la costruzione mentale di rappresentazioni funzionali del sistema con
il quale si interagisce: ciò si applica sia al bambino che all'adulto.
L'\textbf{esperienza empirica} aiuta l'essere umano a costruire materiali concreti di riferimento da "consultare" in relazione alla realtà esterna: questo, per giunta, avviene in modo particolarmente efficace 
e soddifacente quando vi è anche una realizzazione pratica e concreta di questi concetti. 
La teoria costruzionista di Papert pone l'accento non solo sul ruolo attivo dello studente, ma anche sul fatto che la "costruzione" divenga ancora più efficace in un contesto in cui l'individuo produce un risultato per lui 
ricco di significato e di utilità.

In questo modo, come si legge nel libro "\textit{Mindstorms: Children, Computers, and Powerful Ideas}", Seymour Papert propose un nuovo modello educativo in cui strumenti come il computer non sono più visti come "contenitori di informazioni" da cui attingere, ma come strumenti per costruire e manipolare le proprie idee e la propria fantasia.
Inoltre, in un altro suo scritto del 1993 "\textit{Children's Machine – Rethinking school in the age of the computer }", Papert citò il famoso proverbio africano "Dai un pesce ad una persona, e quella persona sarà sazia per un giorno,
insegna ad una persona a pescare, e quella persona si sazierà per tutta la vita.". Il Costruzionismo si basa proprio sull'idea che il bambino impari in modo migliore "pescando" in autonomia le competenze e le informazioni di cui
ha bisogno: l'istituzione scolastica ha il compito di supportarlo moralmente, psicologicamente, materialmente e intellettualmente nei suoi sforzi. 

In particolare, Papert capì che, specialmente per quanto riguararda i giovani studenti, la \textbf{programmazione informatica} può essere l'attività ideale per sviluppare questo tipo
di competenze. Lo studioso, infatti, era convinto che il focus di questo tipo di attività dovesse essere il ragionamento e la costruzione di un algoritmo che risolva
un problema, insieme alla capacità di riconoscere e correggere gli errori. In generale, ritenne poi necessario compiere delle operazioni di semplificazione di alcune istruzioni e concetti informatici, i quali potevano risultare troppo complessi
ma anche non necessari allo scopo. 

Secondo tali premesse, Papert sviluppò un proprio linguaggio, \textbf{LOGO} - il cui nome deriva dal greco antico %TODO
("parola" oppure "pensiero") - , un ambiente di programmazione visuale che permette di controllare una tartaruga, scrivendo
codice e imparando a padroneggiare i concetti di base dell'informatica in modo semplice e divertente. L'ambiente di questo linguaggio, fortemente grafico ma allo stesso tempo semplice, 
è adatto a bambini di tutte le età, permettendo loro non solo di imparare a programmare, ma anche di sfruttare al massimo la loro \textbf{creatività e fantasia}.
LOGO, inoltre, fu costruito in modo tale da permettere di costruire quei procedimenti cognitivi che permettono di individuare errori e inconvenienti e di imparare a costruire un ragionamento 
critico che aiuti a far fronte a questo tipo di situazioni, accettando tutte le soluzioni come possibili e non etichettandone alcune come sbagliate a priori.

Tra le caratteristiche più importanti di LOGO, inoltre, vi è la possibilità di utilizzarlo non soltanto come supporto alle materie matematiche o scientifiche, 
ma anche come \textbf{strumento interdisciplinare} per la creazione di attività di carattere artistico e umanistico.
Il computer diventa in questo modo strumento di espressione e di costruzione di conoscenza, e non solo un mezzo per l'apprendimento di nozioni preconfezionate: inoltre, grazie
all'elaboratore, lo studente può vedere il risultato concreto delle proprie conoscenze tradotte in algoritmi e programmi.
Come anticipato, LOGO permette di controllare una tartaruga. La tartaruga in questione non è altro che un cursore grafico che, quando si muove, lascia il segno 
del proprio percorso dietro di sé grazie ad una penna: in questo modo, muovendosi, la tartaruga può disegnare figure geometriche di ogni tipo.
La "geometria della tartaruga" è un'area didattica creata da Seymour Papert e i suoi colleghi proprio per rendere più semplice e divertente l'apprendimento, in questo caso, della matematica e della geometria.
I comandi base della geometria della tartaruga sono molto semplici: Forward (avanti); Back (indietro); Right (destra); Left (sinistra); Penup (penna su, ovvero non appoggiata sul foglio), Pendown (penna sul foglio). 
A partire da questi semplici comandi si possono ideare problemi e sfide di ogni tipo e di ogni grado di difficoltà, mettendo in campo conoscenze pregresse sulla materia e acquisendone di nuove. 
In ogni caso, LOGO, come tanti altri strumenti di programmazione visuale e non solo, costituisce una preziosa risorsa sia per gli studenti che per gli insegnanti, poiché 
promette un \textbf{nuovo modo di imparare}, sicuramente più attivo e coinvolgente, ma anche più efficace e duraturo.

%TODO: aggiungere Resnick 

\subsubsection{Jeannette Wing}
Il concetto di "pensiero computazionale" è stato poi ripreso e ampliato da \textbf{Jeannette Wing}, informatica teorica e attuale vicepresidentessa di Microsoft Research.

Nel 2006, Wing pubblicò un articolo intitolato "\textit{Computational Thinking}" in cui definì la propria visione di pensiero computazionale; nell'articolo in questione, 
Wing parla di pensiero computazionale come "un approccio universalmente applicabile e un set di skills che tutti noi dovremmo imparare e usare, non soltanto gli informatici". Nello stesso scritto,
fu proposto proprio questo concetto come la quarta abilità
fondamentale da insegnare nelle scuole, insieme alla lettura, alla scrittura e all'aritmetica.

Infatti, secondo Wing, l'informatica non offre soltanto tecnologie straordinarie, ma anche un \textbf{framework intellettuale} per un nuovo modo di pensare, utile a tutti. 
Wing scrisse: "Il pensiero computazionale coinvolge la risoluzione di problemi, la progettazione di sistemi e la comprensione del comportamento umano, attingendo dai concetti
fondamentali del mondo dell'informatica.[\ldots] Il pensiero computazionale è riformulare un problema che sembra difficile in uno che sappiamo già come risolvere, ad esempio
attraverso processi di riduzione, incorporamento, trasformazione o simulazione".\space\textit{Divide et impera}: ecco la prima competenza fondamentale che il pensare in modo computazionale insegna. La 
\textbf{decomposizione di un problema in sottoproblemi più piccoli}, più facili da risolvere, aiuta ad esercitare un certo grado di controllo e ordine sul problema originale, e contemporaneamente, lo rende più 
gestibile e più facilmente comprensibile. Sulla stessa linea, nasce anche la \textbf{capacità di riconoscere e utilizzare pattern}, ovvero
soluzioni a problemi già risolti in passato che possono essere riutilizzate in nuovi contesti. Ma per poter risolvere un problema in modo efficiente, è anche necessario 
\textbf{saper astrarre e generalizzare}, analizzando in modo critico i dati e le informazioni a disposizione, eliminando quelle superflue e concentrandosi su quelle essenziali.
Infine, Wing invita a \textbf{pensare in modo algoritmico}, ovvero a sviluppare una sequenza di passi logici e ben definiti che, se seguiti correttamente, portano alla soluzione del problema.
Si alternano, dunque, momenti in cui si riflette sul problema, si analizzano i dati e si progettano soluzioni, applicando conoscenze pregresse e nuove, a momenti in cui 
si mette in pratica il frutto di questo lavoro mentale, eventualmente scrivendo codice e testando la soluzione proposta, correggendo eventuali errori.
Pensare in modo computazionale, dunque, non ha solo a che vedere con l'essere in grado di 
programmare un computer: significa anche essere in grado di \textbf{pensare a più livelli di astrazione}. Jeannette Wing è convinta, come molti esperti del settore, che il pensiero computazionale 
non sia una competenza riservata soltanto agli informatici e alle persone di scienza: insegnarlo a scuola non vuol dire necessariamente formare futuri programmatori o ingegneri, ma cittadini più
consapevoli e competenti in un mondo sempre più tecnologico e complesso. In questo mondo, è la macchina a dover pensare come l'uomo,
e non il contrario: l'essere umano è creativo e intelligente, la macchina è passiva; è l'utente a dover rendere la macchina interessante per poterla utilizzare al massimo delle sue potenzialità.

\section{Il Micromondo}
Fu Seymour Papert \ref{papert} per primo ad introdurre il concetto di "{micromondo}" nel contesto informatico ed educativo.
Il micromondo di Papert nacque contestualmente all'introduzione del suo linguaggio di programmazione, LOGO, e della relativa protagonista, la tartaruga. L'ambiente di apprendimento della tartaruga fu pensato e realizzato per permettere ai bambini di ragionare su concetti e 
sfide di carattere matematico e geometrico: non a caso, Papert lo aveva soprannominato "provincia di Mathland", come ad indicare un piccolo mondo, confinato e protetto, costruito ad hoc con 
un obiettivo ben preciso, quello di fare scuola in modo assolutamente innovativo.

Secondo Papert, infatti, il micromondo nasce come un ambiente circoscritto, in cui è possibile sperimentare e coltivare le proprie idee con un alto grado di libertà e autonomia, acquisendo 
conoscenze in modo naturale e graduale, proprio come avviene nei primi anni di vita dell'essere umano. Generalmente, la costruzione di micromondi avviene attraverso una serie di fasi e 
principi ben definiti, che contribuiscono a rendere l'ambiente il più efficace e stimolante possibile. 

In primo luogo, come già accennato, i micromondi si basano sempre su un tema ben preciso: il dominio in cui si svolge l'attività deve fare in modo di spostare il focus dell'attenzione degli studenti sul mettere in pratica le conoscenze acquisite e i loro personali modi di pensare. 
Naturalmente, il tema scelto deve essere non solo facilmente comprensibile da chi lo programma, ma deve lasciare spazio alla creatività e alla fantasia. Papert stesso scrisse: "\textit{Low floors, high ceilings}".
\textit{Low floors}, traducibile come "pavimenti bassi", richiama la facilità di accesso e di comprensione che ogni micromondo deve garantire: l'ambiente deve essere alla portata di tutti, 
senza discriminazioni di alcun tipo.
\textit{High ceilings}, ovvero "alti soffitti", invece, si riferisce alla possibilità di oltrepassare i limiti imposti dall'ambiente stesso e di dare spazio alla componente fantasiosa e creativa di ogni studente; in tal 
modo le possibilità di realizzazione del micromondo sono pressoché infinite.
Successivamente l'espressione venne ampliata da Mitchel Resnick, uno degli allievi di Papert e attualmente professore presso il MIT di Boston, con l'aggiunta di "\textit{wide walls}", ovvero "mura larghe".
Resnick si riferiva alla possibilità di estendere il micromondo includendo concetti e sfide sempre nuove e attuali, facendo in modo di poter continuare a raccogliere 
contributi e idee da parte di tutti coloro che hanno accesso ad esso.

In questo modo, si rende possibile la creazione di contesti didattici stimolanti e immersivi, dove \textbf{l'aspetto ludico e quello educativo si fondono in modo armonioso e naturale}, permettendo agli studenti di
sentirsi personalmente coinvolti e motivati.

Tutto ciò, nondimeno, non sarebbe possibile senza il coinvolgimento attivo degli insegnanti, che devono essere in grado sia di gettare le basi per la costruzione del micromondo, sia di 
supportare e guidare l'alunno nel suo percorso di apprendimento, che risulta significativamente più autonomo e personale rispetto a quello tradizionale. Il docente, in questo nuovo modo di fare scuola, 
assume il ruolo di guida e di facilitatore, piuttosto che di dispensatore di nozioni e conoscenze: deve necessariamente lasciare spazio a tutte le idee degli studenti, accettando e valorizzando anche 
il ruolo degli "\textbf{errori}" come parte integrante del processo di apprendimento. 

Non solo: gli studenti possono sperimentare un approccio più attivo e partecipativo rispetto alla didattica frontale.
Il micromondo, infatti, è un ottima occasione di \textit{learn by doing}, in cui i discenti sperimentano e costruiscono per pezzi, in modo
incrementale, le loro idee e i loro ragionamenti, producendo un risultato significativo e soprattutto tangibile. 
In particolare, i risultati più importanti di questo metodo educativo si ritrovano nello sviluppo di una lunga serie di \textbf{competenze trasversali}, o \textit{soft skills}. Da una maggiore capacità di   \textit{problem-solving} e
pensiero critico, a una più sviluppata autonomia e responsabilità, non dimenticando un importante sviluppo nella capacità di collaborazione e di comunicazione se si lavora in gruppo, il micromondo si rivela
un'ottima occasione di crescita e di apprendimento per gli studenti di ogni età.

\section{Il \textit{Coding}}
I concetti emersi nei paragrafi precedenti aprono le porte a un'ulteriore riflessione, quella riguardante gli strumenti e le metodologie corrette per
trasmettere la \textit{forma mentis} del pensiero computazionale.  
L'informatica possiede una dignità intrinseca quale disciplina scientifica fondamentale, caratterizzata da un corpus consolidato di concetti e metodologie indipendenti. 
Inoltre, riveste un valore significativo anche come \textbf{disciplina trasversale}, offrendo un approccio che contribuisce a una più profonda comprensione delle altre discipline.
Le "Indicazioni nazionali per il curricolo della scuola dell'infanzia e del primo ciclo d'istruzione\footnote[1]{\url{gne}}", pubblicate dal MIUR nel 2012, 
sottolineano come la scuola italiana debba evolversi, permettendo alle nuove generazioni di sfruttare le competenze base – che, in passato, erano l'obiettivo primo del processo di alfabetizzazione – 
come strumenti per una miglior comprensione del contesto sociale in cui è inserito lo studente.  

Ken Robinson, nella prefazione allo scritto "\textit{Lifelong Kindergarten - Cultivating Creativity through Projects, Passion, Peers, and Play}" di Mitchel Resnick,
riflette sul fatto che la tecnologia ha sempre assistito l'uomo nell'espansione del proprio corpo e della propria mente.
La capacità umana di \textbf{astrazione e immaginazione} è distintiva e fondamentale per l'evoluzione della società e della cultura. 
Tuttavia, per poter alimentare queste qualità, è necessario disporre di strumenti e materiali 
adatti a tale scopo. Poiché l'informatica consente di costruire rappresentazioni e soluzioni per situazioni anche fisicamente irrealizzabili, limitate solo dalla nostra immaginazione, essa si presenta
come un ambiente potente, utile per esercitare la creatività, assumendo anche un elevato valore educativo. In un contesto in cui le informazioni sono facilmente accessibili e i computer sono sempre più 
abili nel risolvere compiti noiosi e ripetitivi, diventa ancor più cruciale educare gli studenti a trovare \textbf{soluzioni innovative}, anziché concentrarsi sull'applicazione meccanica di procedure 
mnemoniche o standardizzate.
Resnick, a tal proposito, sottolinea che il concetto di creatività viene associato principalmente all'espressione artistica o alla "Creatività con la C maiuscola" di grandi inventori e artisti le
cui opere decorano i musei. La creatività, però, non appartiene solo agli artisti, ma anche agli scienziati che fanno scoperte, ai medici che diagnosticano malattie o agli imprenditori che sviluppano nuovi prodotti 
o strategie di mercato. In modo più generale, la creatività entra in gioco quando si generano idee utili per sé stessi o per gli altri nella vita quotidiana, non necessariamente "rivoluzionarie" a livello
globale, ma sicuramente significative a livello personale. In altre parole, si tratta di una qualità alla portata di tutti, purché sia coltivata fin dalla giovane 
età, ancora meglio se nei contesti educativi e scolastici.

Se si pensa al mondo dell'informatica, dunque, il \textit{coding}, ovvero la scrittura di codice, si configura come uno strumento di straordinaria potenza
nell'ambito dell'educazione digitale e nello sviluppo del pensiero computazionale: si tratta di un profondo tessuto
di competenze che vanno ben oltre la mera scrittura di algoritmi e istruzioni.

In primo luogo, la programmazione agisce come \textbf{catalizzatore del pensiero critico}, 
spingendo gli studenti a frazionare intricati problemi in componenti più gestibili, fornendo loro un approccio strutturato nella risoluzione delle sfide. Quando
inserito in contesti di piccoli gruppi di lavoro, il \textit{coding} favorisce \textbf{la collaborazione e il sinergico lavoro di squadra}, sottolineando l'importanza della condivisione
di idee e della cooperazione per raggiungere obiettivi comuni. La \textbf{componente creativa} del \textit{coding} emerge in modo preminente: risolvere un problema non richiede 
solo la padronanza di conoscenze pregresse, ma anche la capacità di \textbf{immaginare e progettare soluzioni}, valutandone attivamente i pro e i contro e selezionando
la strategia più idonea. In questo contesto, il \textit{coding} diventa un veicolo per l'espressione personale e il confronto con diverse prospettive, aprendo a un ventaglio
infinito di possibilità creative. Infine, al di là delle competenze tecniche specifiche, il \textit{coding} prepara gli studenti per un futuro digitale in continua evoluzione,
potenziando la loro \textbf{alfabetizzazione tecnologica} e aprendo porte a innumerevoli opportunità nel vasto campo della tecnologia. In sintesi, il \textit{coding} non solo istruisce
sul linguaggio della programmazione, ma plasma \textbf{menti agili, creative e collaborative}, pronte a navigare le sfide della società digitale moderna.

Tuttavia, l'introduzione delle discipline informatiche nel contesto educativo non è un'operazione immediata né tantomeno semplice.
Al giorno d'oggi vi sono ancora pregiudizi e critiche che circondano l'utilizzo - spesso eccessivo e improprio - della tecnologia da parte dei giovani.

Una possibile sfida, ad esempio, riguarda l'associazione concettuale che vede l'informatica come un'attività principalmente giocosa e ludica: se da un lato è vero che il \textit{coding} può essere un'attività divertente e stimolante,
dall'altro è importante sottolineare che la programmazione è una disciplina seria e complessa, che richiede molto impegno e dedizione. L'aspetto ludico che il \textit{coding} può suscitare in chi lo utilizza 
non deve essere confuso con la sua reale natura, ovvero, come già detto, quella di uno strumento potente e versatile per lo sviluppo di competenze trasversali e per l'acquisizione di un pensiero critico e creativo.

Un altro possibile ostacolo, diametralmente opposto per quanto riguarda ciò di cui si è parlato poc'anzi, risiede nei luoghi comuni e nei pregiudizi che ancora oggi circondano la disciplina.
Idee sbagliate e stereotipate vedono attività quali programmare e ragionare da informatici come appannaggio di poche persone che hanno quello che gli americani
chiamano il "gene del nerd” (\textit{Geek gene}) %citooo
spesso associato a personalità asociali e competitive e, più in generale, ad un target prevalentemente maschile. Proprio a causa di queste idee, gli esperti del settore 
tendono a sottolineare in modo deciso che l'obiettivo non è quello di insegnare informatica, nonostante non vi sia nulla di immeritevole in tale disciplina, ma si cerca di spostare l'attenzione 
sul \textit{coding} come strumento per lo sviluppo di competenze trasversali e per l'acquisizione di un pensiero critico e creativo.

In ultimo, vi sono anche delle difficoltà di natura pratica, legate alla formazione degli insegnanti e alla disponibilità di strumenti e risorse adeguate.
La formazione degli insegnanti, infatti, è un aspetto cruciale per garantire il successo di un'iniziativa di questo tipo: non solo è necessario che gli insegnanti siano in grado di padroneggiare
le competenze tecniche necessarie per insegnare il \textit{coding}, ma è anche importante che siano in grado di integrare queste competenze in un contesto educativo più ampio,
in modo da garantire che il \textit{coding} non sia un'attività isolata, ma un'attività che si integri in modo armonioso con le altre materie e discipline.
Per quanto riguarda, invece, la possibilità di avere accesso a tecnologie e strumenti utili per questa tipologia di insegnamento, bisogna considerare che non tutte le scuole
del mondo sono in grado di permettersi di acquistare i materiali necessari. In aree più svantaggiate, ad esempio, dove è già difficile garantire l'accesso a materiali didattici di base,
l'idea di introdurre il \textit{coding} come disciplina può sembrare un lusso eccessivo. Tuttavia, moltissimi studi hanno dimostrato che l'uso di strumenti tecnologici non è una 
\textit{conditio sine qua} per insegnare la programmazione informatica: esistono infatti metodologie molto efficaci e, al contempo assolutamente economiche, che riescono a trasmettere il pensiero computazionale
anche senza l'uso del computer. Il \textit{coding} può essere insegnato anche attraverso attività di \textit{role-playing}, di \textit{problem-solving} e di \textit{game-based learning} dette \textit{unplugged}. 
Un esempio di questi studi è stato compiuto da Arinchaya Threekunprapa e Pratchayapong Yasri, che hanno dimostrato come l'uso di \textit{unplugged coding} usando dei \textit{flowblocks} abbia portato a risultati
ottimi per quanto riguarda la promozione del pensiero computazionale tra studenti della scuola secondaria. Il paper, pubblicato nel 2020, mette in luce come, anche utilizzando delle semplici
forme di carta - rappresentanti i diversi blocchi di codice e i relativi significati algoritmici - si possa ottenere un apprendimento significativo e duraturo. L'attività proposta era \textit{game based}, ovvero
posta in un contesto ludico e divertente, e ha portato a risultati molto positivi, sia in termini di apprendimento che di coinvolgimento degli studenti. I partecipanti, lavorando in coppia, 
non solo hanno dovuto collaborare e comunicare tra loro per risolvere le diverse sfide, ma hanno avuto anche un supporto morale e psicologico nel momento in cui si sono trovati ad affrontare 
sfide sempre più difficili, incontrando anche errori e fallimenti. Il confronto, però, ha reso la fase del \textit{debugging} molto più semplice e veloce, e, insieme anche ad una buona dose di competitività
tra i vari gruppi, tutti i partecipanti sono riusciti a portare a termine con successo le sfide proposte. La fase finale della sperimentazione di Threekunprapa e Yasri ha visto i partecipanti - che non 
avevano alcuna esperienza pregressa di programmazione - migliorare significativamente rispetto al test iniziale.


