\chapter{Analisi critica degli strumenti utilizzati}\label{chap:analysis}
\section{Snap!}
Snap! Build Your Own Blocks è un ambiente di programmazione visuale che si basa sulla sintassi dei blocchi, progettato per rendere l'apprendimento della programmazione accessibile e divertente. Sviluppato dal MIT Media Lab, è una variante di Scratch, un altro ambiente di programmazione visuale. Di seguito, fornirò una descrizione approfondita di Snap! Build Your Own Blocks, esplorando diverse caratteristiche e aspetti del linguaggio di programmazione visuale.

1. Interfaccia utente:

Canvas: L'interfaccia principale di Snap! è un canvas vuoto su cui gli utenti posizionano e collegano i blocchi.
Blocchi: I blocchi sono gli elementi fondamentali del linguaggio. Ogni blocco rappresenta un comando o una funzione specifica.
2. Tipi di Blocchi:

Blocchi di Comandi: Questi blocchi eseguono azioni specifiche, come "muovi avanti di 10 passi" o "gira a sinistra di 90 gradi".
Blocchi di Controllo: Gestiscono la sequenza di esecuzione del programma con strutture come "ripeti" o "se".
Blocchi di Operatori: Effettuano operazioni matematiche, logiche o di confronto.
Blocchi di Variabili: Consentono agli utenti di creare e utilizzare variabili nelle loro programmazioni.
3. Funzionalità Principali:

Modularità: Gli utenti possono creare propri blocchi personalizzati per organizzare e riutilizzare il codice.
Costume e Sprite: Gli oggetti visivi (Sprite) possono avere diverse apparenze (Costume), e gli utenti possono controllarli tramite il codice.
Eventi: Gli script possono essere attivati da vari eventi, come clic del mouse, tastiera o in risposta a condizioni specifiche.
Sensori e Variabili Globali: Snap! supporta sensori virtuali (posizione del mouse, tempo) e variabili globali per la memorizzazione di informazioni a livello di progetto.
4. Programmazione di Livello Avanzato:

Procedure Definite dall'Utente: Gli utenti possono creare procedure personalizzate, raggruppando blocchi e dando loro un nome.
Ricorsione: Snap! supporta la ricorsione, consentendo agli utenti di definire funzioni che chiamano se stesse.
Manipolazione delle Liste: Gli utenti possono lavorare con liste e array per gestire dati in modo strutturato.
5. Esportazione e Condivisione:

Esportazione del Codice: Il codice creato in Snap! può essere esportato in vari formati, inclusi JavaScript e Python, per essere eseguito al di fuori dell'ambiente.
Condivisione Online: Gli utenti possono condividere i propri progetti online attraverso il sito web di Snap!, incoraggiando la condivisione e la collaborazione.
6. Comunità e Supporto:

Forum e Risorse: Snap! dispone di una comunità attiva con forum online e risorse di apprendimento per supportare gli utenti.
Documentazione: Una documentazione completa fornisce informazioni dettagliate su ogni aspetto del linguaggio.
7. Aspetti Pedagogici:

Apprendimento Interattivo: L'approccio visuale rende l'apprendimento della programmazione intuitivo ed interattivo, adatto a un pubblico giovane e a principianti.
Promozione del Pensiero Computazionale: Snap! mira a sviluppare il pensiero computazionale, incoraggiando la risoluzione di problemi e la creatività attraverso la programmazione.
In conclusione, Snap! Build Your Own Blocks offre un ambiente di programmazione visuale completo, ricco di funzionalità e progettato per facilitare l'apprendimento e la pratica della programmazione in modo divertente e coinvolgente.
\section{CoSpaces Edu}
\section{Croquet}
\section{Confronto tra strumenti}